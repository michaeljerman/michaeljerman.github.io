% Created 2018-05-21 Mon 07:33
% Intended LaTeX compiler: pdflatex
\documentclass[10pt]{beamer}
\usepackage[utf8]{inputenc}
\usepackage[T1]{fontenc}
\usepackage{graphicx}
\usepackage{grffile}
\usepackage{longtable}
\usepackage{wrapfig}
\usepackage{rotating}
\usepackage[normalem]{ulem}
\usepackage{amsmath}
\usepackage{textcomp}
\usepackage{amssymb}
\usepackage{capt-of}
\usepackage{hyperref}
\usetheme{Boadilla}
\author{ECON 420: Game Theory}
\date{Spring 2018}
\title{Uncertainty II}
\usecolortheme{seagull}
\usefonttheme[onlylarge]{structurebold}
\usefonttheme[onlymath]{serif}
\setbeamerfont*{frametitle}{size=\normalsize,series=\bfseries}
\setbeamertemplate{navigation symbols}{}
\setbeamertemplate{itemize item}[triangle]
\setbeamertemplate{footline}{}
\setbeamertemplate{enumerate items}[default]
\hypersetup{
 pdfauthor={ECON 420: Game Theory},
 pdftitle={Uncertainty II},
 pdfkeywords={},
 pdfsubject={},
 pdfcreator={Emacs 25.2.2 (Org mode 9.1.6)}, 
 pdflang={English}}
\begin{document}

\maketitle

\begin{frame}[label={sec:org6bc779d}]{}
\alert{Market for Lemons (Akerlof, 1970)}
\begin{itemize}
\item Two types of used cars: high quality and low quality ("lemons")
\item Buyers cannot directly observe car quality
\begin{itemize}
\item Willing to pay \$6,000 for low quality
\item Willing to pay \$16,000 for high quality
\end{itemize}
\item Sellers know quality of their car
\begin{itemize}
\item Value of high quality: \$12,500
\item Value of low quality: \$3,000
\end{itemize}
\end{itemize}
\end{frame}

\begin{frame}[label={sec:org272ebec}]{}
\alert{Buyers}
\begin{itemize}
\item Single market price for both cars \(p\)
\begin{itemize}
\item Buyers can't observe quality, so only one market
\end{itemize}
\item Suppose buyers can observe fraction of high quality cars in the market \(f\)
\item Buyers will purchase a used car if \(EV > p\)
\end{itemize}
\end{frame}

\begin{frame}[label={sec:orgcdfb299}]{}
\alert{Sellers}
\begin{itemize}
\item High quality owners sell if \(p>12,500\)
\item Low quality owners sell if \(p>3000\)
\item When will high-quality cars be bought and sold on the market?
\end{itemize}
\end{frame}

\begin{frame}[label={sec:org59be174}]{}
\alert{Market for high quality cars}
\begin{itemize}
\item What happens if \(f\) falls below the critical value?
\item Consumers would be willing to pay \(p>12,500\) (if they were assured of getting high quality car)
\item But sellers unable to find buyers at that price
\item No high quality cars will be bought or sold!
\begin{itemize}
\item \(f=0\)
\end{itemize}
\item Market price: \(3,000<p<6,000\)
\end{itemize}
\end{frame}

\begin{frame}[label={sec:org30540b9}]{}
\alert{Market failure}
\begin{itemize}
\item Sellers willing to sell high quality for \(p>12,500\)
\item Buyers willing to buy high quality for \(p<16,000\)
\item Market equilibrium is \emph{inefficient}
\begin{itemize}
\item Buyers \emph{and} sellers could be made better off without making anyone worse off
\end{itemize}
\end{itemize}
\end{frame}

\begin{frame}[label={sec:orgbb1bd38}]{}
\alert{Adverse selection}
\begin{itemize}
\item Market for lemons is an example of \emph{adverse selection}
\item With imperfect information, one side must form expectations
\item This can make \(p\) low enough that some high-quality goods (or services) exit the market
\item This reduces EV, further reducing price
\item Only products that remain in market are low quality
\end{itemize}
\end{frame}

\begin{frame}[label={sec:org994f94a}]{}
\alert{Example: Health insurance}
\begin{itemize}
\item Health insurance providers do not know consumers' health as well as the individual
\item Offer a single price to "similar" consumers
\item Healthiest consumers may find price to be too high (expected utility greater without insurance)
\item This reduces the average health of remaining consumers
\item Insurance companies raise price to account for higher risk
\begin{itemize}
\item Process repeats
\end{itemize}
\item "Death spiral" -- market collapses
\begin{itemize}
\item Rationale behind "individual mandate" and universal systems
\end{itemize}
\end{itemize}
\end{frame}

\begin{frame}[label={sec:orgc98faaf}]{}
\alert{Signalling}
\begin{itemize}
\item How do we overcome uncertainty and adverse selection?
\item "Cheap talk" insufficient, no cost to lying about quality
\item Quality signals must impose a cost on the player signalling in order to prevent "cheaters"
\item Example: Product warranties
\begin{itemize}
\item Used car dealers sometimes offer promises to repair a car if it breaks down
\item Unlikely to offer warranty for lemons
\end{itemize}
\item Example: Product design and marketing
\begin{itemize}
\item Producers of high-quality products can "afford" fancy design and marketing
\end{itemize}
\end{itemize}
\end{frame}

\begin{frame}[label={sec:org36ddfc9}]{}
\alert{Screening}
\begin{itemize}
\item Why do employers value some majors higher than others?
\item In general, degrees that are harder result in higher wages
\item Employers know that "low quality" workers are unlikely to take hard classes
\begin{itemize}
\item High quality workers may have an easier time, less of an investment
\end{itemize}
\item This separates high-quality from low-quality workers
\item Major choice is (partly) a \emph{signal} to employers
\end{itemize}
\end{frame}

\begin{frame}[label={sec:orga219b72}]{}
\alert{Separating vs pooling equilibrium}
\begin{itemize}
\item Screening costs must be high enough to prevent low-quality players from cheating
\item But screening costs must be \emph{low} enough to make it worth it for high-quality players to join
\item If the screening costs are too high or too low, then all players will \emph{pool} together and be indistinguishable
\item Examples: California food labeling laws vs "certified organic" labels
\end{itemize}
\end{frame}
\end{document}