% Created 2018-04-09 Mon 07:20
% Intended LaTeX compiler: pdflatex
\documentclass[10pt]{beamer}
\usepackage[utf8]{inputenc}
\usepackage[T1]{fontenc}
\usepackage{graphicx}
\usepackage{grffile}
\usepackage{longtable}
\usepackage{wrapfig}
\usepackage{rotating}
\usepackage[normalem]{ulem}
\usepackage{amsmath}
\usepackage{textcomp}
\usepackage{amssymb}
\usepackage{capt-of}
\usepackage{hyperref}
\usetheme{Boadilla}
\author{ECON 420: Game Theory}
\date{Spring 2018}
\title{Introduction}
\usecolortheme{seagull}
\usefonttheme[onlylarge]{structurebold}
\usefonttheme[onlymath]{serif}
\setbeamerfont*{frametitle}{size=\normalsize,series=\bfseries}
\setbeamertemplate{navigation symbols}{}
\setbeamertemplate{itemize item}[triangle]
\setbeamertemplate{footline}{}
\hypersetup{
 pdfauthor={ECON 420: Game Theory},
 pdftitle={Introduction},
 pdfkeywords={},
 pdfsubject={},
 pdfcreator={Emacs 25.2.2 (Org mode 9.1.6)}, 
 pdflang={English}}
\begin{document}

\maketitle

\begin{frame}[label={sec:org8f2a5e4}]{}
\alert{Instructor}
\begin{itemize}
\item Michael Jerman (call me Michael)
\item Office: 342 Bexell
\item Email: michael.jerman@oregonstate.edu
\item Office hours:
\begin{itemize}
\item M/W: 10:30am-11:30am
\item T/Th: 8:30am-9:30am
\end{itemize}
\end{itemize}
\end{frame}

\begin{frame}[label={sec:org88473fa}]{}
\alert{Textbook}
\begin{itemize}
\item \emph{Games of Strategy}, Dixit, Skeath, Reiley
\begin{itemize}
\item 4th edition
\end{itemize}
\item \href{https://www.amazon.com/Games-Strategy-Fourth-Avinash-Dixit/dp/0393919684/}{Amazon}
\begin{itemize}
\item \$120 new
\item \$90 used
\item \$36 rent (electronic version)
\end{itemize}
\item \href{http://verbacompare.osubeaverstore.com/compare/2018-Spring\_\_ECON\_\_420\_\_001}{Bookstore}
\begin{itemize}
\item \$132 new
\item \$99 used
\end{itemize}
\end{itemize}
\end{frame}

\begin{frame}[label={sec:org73cfe75}]{}
\alert{Prerequisites}
\begin{itemize}
\item Econ 311 or Econ 411 (intermediate micro)
\end{itemize}
\end{frame}

\begin{frame}[label={sec:org0cb571f}]{}
\alert{Canvas}
\begin{itemize}
\item Official course website
\item You are responsible for any and all information posted to Canvas
\item Contact canvas@oregonstate.edu if you have any issues
\end{itemize}
\end{frame}

\begin{frame}[label={sec:orgd9cdc01}]{}
\alert{Participation}
\begin{itemize}
\item What's game theory without games?
\item We will frequently do in-class exercises and games
\item Participation will be recorded
\item 10\% of your grade just for showing up!
\end{itemize}
\end{frame}

\begin{frame}[label={sec:org033cd39}]{}
\alert{Homework}
\begin{itemize}
\item Four graded homework assignments, posted on Canvas
\item Mix of problems from the text and other problems
\item Work in groups!
\begin{itemize}
\item Each person must submit their own assignment
\end{itemize}
\item Graded primarily on effort and completion
\begin{itemize}
\item 25\% of your grade just for doing the homework!
\end{itemize}
\end{itemize}
\end{frame}

\begin{frame}[label={sec:org6211e8e}]{}
\alert{Exams}
\begin{itemize}
\item Midterm: Wednesday, May 2 
\begin{itemize}
\item Week 5
\end{itemize}
\item Final: Friday, June 15 at 7:30am (!)
\begin{itemize}
\item Note that the university might change the final exam date
\end{itemize}
\item No makeup exams (see syllabus)
\end{itemize}
\end{frame}

\begin{frame}[label={sec:orgc34512c}]{}
\alert{Grade}
\begin{center}
\begin{tabular}{ll}
Participation & 10\%\\
Homework & 25\%\\
Midterm & 30\%\\
Final & 35\%\\
\end{tabular}
\end{center}
\end{frame}

\begin{frame}[label={sec:orgb410fbf}]{}
\alert{Student conduct}
\begin{itemize}
\item Student's are bound by the university's \href{http://studentlife.oregonstate.edu/sites/studentlife.oregonstate.edu/files/code\_of\_student\_conduct.pdf}{Code of Student Conduct}
\item Any incidence of academic misconduct will result in a grade of "F" for the course 
\begin{itemize}
\item Additional sanctions may be imposed by the university
\end{itemize}
\end{itemize}
\end{frame}

\begin{frame}[label={sec:org0d89b84}]{}
\alert{Important dates}
(Subject to change)
\begin{center}
\begin{tabular}{ll}
Wednesday, April 18 & Homework 1\\
Monday, April 30 & Homework 2\\
Wednesday, May 2 & Midterm\\
Wednesday, May 16 & Homework 3\\
Wednesday, June 6 & Homework 4\\
Friday, June 15 & Final exam\\
\end{tabular}
\end{center}
\end{frame}

\begin{frame}[label={sec:org3a9827f}]{}
\alert{Half the average game}
\begin{itemize}
\item Take out a blank piece of paper and write your name on the top
\item Next pick a number between 0 and 100 (don't write it just yet)
\begin{itemize}
\item We will record all of the numbers and calculate the average
\item The winner will be the person whose chosen number is \emph{one half} the class average
\end{itemize}
\item Once you've decided, write your number on the paper
\item Trade papers with someone else
\end{itemize}
\end{frame}

\begin{frame}[label={sec:org37b4448}]{}
\alert{"Standard" economics (econ 311)}
\begin{itemize}
\item Agents have preferences over consumption bundles, choose bundle that optimizes their utility
\item Generally consider prices and income to be \emph{exogenous}
\item Agents' consumption choices don't affect the choices of other agents
\item Firms maximize profits subject to constraints
\begin{itemize}
\item Perfect competition: Firms are \emph{price takers} -- individual production decisions don't affect prices
\end{itemize}
\item These choices are called \emph{decisions}: isolated choices that individual agents make given objectives and constraints
\end{itemize}
\end{frame}

\begin{frame}[label={sec:org65c5b49}]{}
\alert{Game theory}
\begin{itemize}
\item Game theory is the study of how agents make choices in environments where the choices of others affects their outcomes \emph{and} their choices
\item Examples:
\begin{itemize}
\item Interactions with family and friends
\item Business decisions
\item Athletic competition
\item Board games
\item Political campaigns
\item Diplomacy
\item Warfare
\item Etc, etc, etc
\end{itemize}
\end{itemize}
\end{frame}

\begin{frame}[label={sec:org5aea281}]{}
\alert{Example: Bertrand competition}
\begin{itemize}
\item Two firms selling perfect substitutes
\item Consumers perfectly observe the prices charged by each firm
\item No transportation cost -- consumers only buy from cheaper firm
\end{itemize}
\end{frame}

\begin{frame}[label={sec:org1df1ff7}]{}
\begin{center}
\includegraphics[height=.5\textwidth]{./img/gas.png}
\end{center}
\end{frame}

\begin{frame}[label={sec:org55489b5}]{}
\alert{Firm choices}
\begin{center}
\begin{tabular}{rl}
76 & Mobil\\
\$3.30 & \$3.20\\
\end{tabular}
\end{center}
\begin{itemize}
\item Where will customers go?
\item What will 76 do?
\item What will Mobil do in response?
\end{itemize}
\end{frame}

\begin{frame}[label={sec:orgdded69e}]{}
\alert{Strategic interaction}
\begin{itemize}
\item Game-theoretic situations differ from decision-theoretic (Econ 311) situations because they are \emph{strategic}
\item When playing a game, a player must consider the other player's preferences when making their choices
\begin{itemize}
\item But the player must \emph{also} recognize that the other player is considering their preferences as well
\begin{itemize}
\item The first player must also recognize that the other player recognizes that player 1 recognizes that the other player is considering player 1's preferences
\begin{itemize}
\item \ldots{}
\begin{itemize}
\item \ldots{}
\end{itemize}
\end{itemize}
\end{itemize}
\end{itemize}
\end{itemize}
\end{frame}

\begin{frame}[label={sec:org8d46ce8}]{}
\alert{Example}
\href{https://www.youtube.com/embed/rMz7JBRbmNo}{https://www.youtube.com/embed/rMz7JBRbmNo}
\end{frame}

\begin{frame}[label={sec:org90d1569}]{}
\alert{Example}
\href{https://www.youtube.com/embed/p3Uos2fzIJ0}{https://www.youtube.com/embed/p3Uos2fzIJ0}
\end{frame}

\begin{frame}[label={sec:org6e171c1}]{}
\alert{Example: Nim}
\begin{itemize}
\item At the start of the game there are two rows of lines (which represent matchsticks, coins, rocks\ldots{}).  For today, the rows start with three and four lines respectively:
\end{itemize}

\begin{align*}
|~|~|~\phantom{|} \\
|~|~|~|
\end{align*}

\begin{itemize}
\item On a player’s turn the player chooses one of the rows and removes (or scratches out) any number of lines as long as they are in that same row. At least one line must be removed per turn.
\item Turns alternate until the last line is removed.
\item The player who removes the last line of all wins.
\end{itemize}
\end{frame}


\begin{frame}[label={sec:org9f9762d}]{}
\alert{Nim}
\begin{itemize}
\item Does either player have an advantage?
\item What are the optimal decisions?
\end{itemize}
\end{frame}

\begin{frame}[label={sec:orgd3a0d3d}]{}
\alert{Split the extra-credit points}
\begin{itemize}
\item Get a blank sheet of paper and write your full name at the top
\item Now choose the amount of extra credit that you would like to receive on the midterm exam (as a percentage)
\item Bring your paper to me on your way out
\item I will add up all of the points chosen
\begin{itemize}
\item If the total number of points is \emph{less} than the number of people in the class, then each of you will get your chosen amount of extra credit on your midterm
\item If the total number of points is \emph{greater} than the number of people in the class, then \emph{nobody} gets any extra credit
\end{itemize}
\item You can communicate with each other if you'd like (be respectful!)
\end{itemize}
\end{frame}
\end{document}